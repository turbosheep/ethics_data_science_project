\documentclass[11pt]{article}

\title{Detecting ``Fake News" on Facebook}
\author{Hannah Eyre\\
	Zane Zakraisek}
\usepackage[margin=1.0in]{geometry}


\begin{document}
\maketitle

The term ``Fake News" gained popularity during the United States' 2016 Presidential Election to describe a rapidly spreading phenomena of news articles deliberately spreading false information and hoaxes, often through attention grabbing headlines or headlines that resemble legitimate sources \cite{guardian}. It became particularly notorious on social media sites and Facebook in particular, where the top 20 articles from hoax sites and hyperpartisan blogs garnered more user interaction between August 1st and election day on November 8th than the top 20 articles from a variety of established news sources such as {\it The New York Times}, {\it Washington Post}, {\it Business Insider}, and Fox News \cite{buzzfeed}.

FactCheck.org, part of the Annberg Public Policy Center at the University of Pennsylvania, breaks down how an individual can identify fake news into eight parts \cite{factcheck}:
\begin{enumerate}
\item Consider whether the source is credible.
\item Read beyond the headline.
\item Check whether the author is credible/real.
\item Check whether the article is recent.
\item Check whether it is a joke/satire.
\item Consider your own biases and how they affect your judgment.
\item Check supporting sources (if any) and make sure they abide by the same rules.
\item Ask experts or fact-checking sites.
\end{enumerate}

Rapidly spreading fake news articles have a range of consequences, one instance culminating in a gunman shooting his gun inside a Washington, DC pizza parlor over allegations of child pornography and child sex abuse ring centered around John Podesta, Hillary Clinton's 2016 campaign manager, in a conspiracy called ``Pizzagate" \cite{pizzagate}. Various politicians and government agencies in the United States and internationally have voiced opinions on what qualifies as fake news and what do do about it, but no consensus has been reached. Facebook was initially reluctant to admit there was any problem with fake news on users news feeds, however Facebook's CEO Mark Zuckerberg has since released a statement describing how they plan to deal with fake news in the future, including renaming the term ``false news" \cite{zuckerberg}.

As an increasingly global and hotly contested issue, we would like to to explore what responsibility Facebook has in regards to these eight points. We will discuss whether they have a responsibility to develop tools to detect fake news based off these guidelines and, if these tools exist, whether they should be used to remove content from the site.

\paragraph{}
When analyzing any issue, it's important to define what framework you're working in, along with what premisses your arguments are based on. During our analyzation of the Facebook's role in censorship, we'll be working under a utilitarian framework. In particular, we'll be using a utilitarian framework designed to minimize the harm. We have chosen this particular form of utilitarianism for a few different reasons. If you examine the role that fake news has played in the numerous events mentioned, it's evident that the outcome is negative in a large majority of occurrences. It's true that news in general on Facebook has the potential to result in positive utility among its readers, but as we've seen, more often than not the potential negatives don't come close to outweighing the corresponding positives. As a result, while it's important to attempt to maintain the positive utility brought about by Facebook's news feeds, we believe that it's far more important to try to minimize the corresponding negative utility.

\paragraph{}
As we address each of these eight topics, the format shall be as follows. We'll first examine the outcome if Facebook chose not to implement this fake news filtering mechanism. Next, we'll examine the outcome if Facebook chose to implement the mechanism. Lastly, we'll weigh the different utilities and select the one that we feel minimizes the negative utility. In addition to this, we'll also be examining whether or not the method is even technically feasible to implement. The final outcome for each point will be a combination of the two points. 

\paragraph{Credible sources: Zane}
\paragraph{Headline vs Body: Hannah}
\paragraph{Author Credibility: Zane}
\paragraph{Age of Article: Hannah}
\paragraph{Article Genre: Hannah}
\paragraph{Interpretation Bias: Hannah}
\paragraph{Article Sources: Zane}
\paragraph{Cross Reference: Zane}

\paragraph{Final utility based on points above}

\bibliographystyle{apalike}
\bibliography{proposal_bibliography.bib}


\end{document}
