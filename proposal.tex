\documentclass[11pt,letterpage]{article}

\title{Detecting ``Fake News" on Facebook}
\author{Hannah Eyre\\
	Zane Zakraisek}

\begin{document}
\maketitle

The term ``Fake News" gained popularity during the United States' 2016 Presidential Election to describe a rapidly spreading phenomena of news articles deliberately spreading false information and hoaxes, often though attention grabbing headlines that garner clicks. It became particularly notorious on social media sites, Facebook in particular, where many fake news articles gathered more attention than real news stories. 

The International Federation of Library Associations and Institutions (IFLA), a partner of the United Nations Educational, Scientific and Cultural Organization (UNESCO), breaks down how an individual can identify fake news into eight parts:
\begin{enumerate}
\item Consider whether the source is credible
\item Read beyond the headline
\item Check whether the author is credible/real
\item Check whether the article is recent
\item Check whether it is a joke/satire
\item Consider your own biases and how they affect your judgement
\item Check supporting sources (if any) and make sure they abide by the same rules
\item Ask experts or fact-checking sites
\end{enumerate}

We want to ask what responsibility Facebook has in regards to these eight points. Whether they have a responsibility to develop tools to detect fake news based off these guidelines and, if these tools exist, whether they should be used to remove content from the site.

\end{document}